\chapter{硬件平台---NEMU}
\section{前言}
编写虚拟机的第一个任务就是去实现其硬件设施。根据冯诺伊曼计算机的思想,一个完整的
计算设备需要有运算器、控制器、存储器、输入设备和输出设备,而本章介绍的NEMU就是
去实现这五大部件。

在今年,南京大学的NEMU项目进行了扩展,使NEMU提供了三种指令集架构可供选择。一个指令
集架构约定了指令的编码方式以及运算器和控制器的解码及执行方式。在NEMU中提供了三种
指令集架构,分别是:\emph{x86},\emph{mips32}和\emph{riscv32}。在这里我选择的是
日常使用中最常见的\arch 架构。

\section{系统设计与实现}
下面我们就正式进入冯诺伊曼机NEMU的设计与实现。由于NEMU源文件较多,为了能够更好的
叙述设计和实现,这里采用自顶向下的方式来阐述。

\subsection{NEMU总体架构}
由于NEMU模拟器是一个冯诺伊曼机,所以其整体架构遵循冯诺伊曼机。NEMU的总体架构图
如图\ref{fig:nemu-arch}所示。

\vspace{5pt}
\hustfigure[0.85\textwidth]{figure/nemu-arch.pdf}{NEMU总体架构图}{fig:nemu-arch}

\subsection{框架代码结构}
\dirtree{%
  .1 nemu/.
  .2 include/.
  .3 cpu/.
  .3 device/.
  .3 memory/\DTcomment{内存访问有关}.
  .3 monitor/\DTcomment{监视器有关}.
  .3 rtl/\DTcomment{通用rtl指令定义}.
  .3 common.h\DTcomment{公用头文件}.
  .3 debug.h.
  .3 macro.h.
  .3 nemu.h.
  .2 src/.
  .3 cpu/\DTcomment{CPU执行有关}.
  .3 device/\DTcomment{IO设备实现}.
  .3 isa/\DTcomment{指令集架构封装}.
  .4 mips32/\DTcomment{mips32指令集}.
  .4 riscv32/\DTcomment{riscv32指令集}.
  .4 x86/\DTcomment{x86指令集}.
  .3 memory/\DTcomment{内存访问实现}.
  .3 monitor/.
  .4 debug/\DTcomment{调试器实现}.
  .5 expr/\DTcomment{表达式解析{\color{red}(这里与给定的框架代码不同)}}.
  .6 def.h\DTcomment{表达式解析有关函数定义}.
  .6 lex.l\DTcomment{表达式解析词法规则定义}.
  .6 parser.y\DTcomment{表达式解析语法规则定义}.
  .5 log.c\DTcomment{Log信息输出}.
  .5 ui.c\DTcomment{监视器交互命令实现}.
  .5 watchpoint.c\DTcomment{监视点实现}.
  .4 diff-test/.
  .4 cpu-exec.c.
  .4 monitor.c.
  .3 main.c.
  .2 tools/\DTcomment{测试及调试用工具}.
  .2 Makefile.
  .2 Makefile.git\DTcomment{git版本控制相关}.
  .2 runall.sh\DTcomment{一键测试脚本}.
}

\subsection{NEMU执行流}
为了能够了解NEMU的工作方式,我们来看看NEMU整体的一个执行流程。

进入\file{nemu/src/main.c}文件,能够看到里面定义了\code{main()}函数。在\code{main()}
函数中只有两行,第一行调用\code{init\_monitor()}函数对NEMU进行各项初始化,并根据
调用参数来判断本次程序运行是否是批处理模式。第二行调用\code{ui\_mainloop()}函数。
\code{ui\_mainloop()}函数在\file{nemu/src/monitor/debug/ui.c}中定义,该函数是
监视器与用户进行IO交互的主函数。在该函数中首先判断程序是否是批处理模式,若是
批处理模式则直接运行在命令行中指定的程序,不会出现与用户的交互界面。若不是批处理
模式则会进行循环,在循环体中首先等待用户的命令,之后根据用户所输入的命令调用相应的
处理函数。

由此我们可以得到NEMU的总体流程图,如图\ref{fig:nemu-flowchart}所示。

\begin{figure}[!htbp]
\centering
\begin{autoflow}
begin
初始化NEMU
if (批处理模式?)
{
  运行程序
}
else
{
  input 命令
  while (非终止命令?)
  {
    执行命令
    input 命令
  }
}
end
\end{autoflow}
\caption{NEMU整体流程图}\label{fig:nemu-flowchart}  
\end{figure}

在NEMU的交互界面中,一共设定了九个命令,这些命令对应的字符串以及其含义如表\ref{tab:nemu-cmd}
所示。

\begin{table}[!htpb]
  \centering
  \caption{NEMU调试器指令}
  \label{tab:nemu-cmd}
  \begin{tabular}{ccc}
    \toprule
    \textbf{命令名称}& \textbf{命令描述}      & \textbf{处理函数}        \\
    \midrule
    help      & 显示帮助        & \code{cmd\_help()}\\
    c         & 继续运行程序    & \code{cmd\_c()}   \\
    q         & 退出NEMU        & \code{cmd\_q()}   \\
    p expr    & 打印表达式值    & \code{cmd\_p()}   \\
    info r    & 打印寄存器信息  & \code{cmd\_info()}\\
    info w    & 打印监视点信息  & \code{cmd\_info()}\\
    w         & 添加监视点      & \code{cmd\_w()}   \\
    d         & 删除监视点      & \code{cmd\_d()}   \\
    s         & 单步执行        & \code{cmd\_s()}   \\
    \bottomrule
  \end{tabular}
\end{table}

在这些命令中,最重要的就是\emph{c}命令,该命令使程序继续运行,在没有设置监视点的
情况下输入\emph{c}命令会使程序一直执行到结束为止。下面我们就来看看\emph{c}命令
的实现。

\subsection{NEMU内程序运行流程}
\emph{c}命令的执行函数是\code{cmd\_c()},该函数在\file{nemu/src/monitor/debug/ui.c}
中定义,该函数的函数体只有一行,即\code{cpu\_exec(-1)}。这个函数在文件
\file{nemu/src/monitor/cpu-exec.c}中定义,该函数有一个参数,这个参数代表了要执行
的虚拟机程序的指令条数。在\code{cmd\_c()}中传入的参数值是-1,由于参数值是按照无符号
数来解析的,所以-1会被认为是最大的整数,也就是说想让程序一直执行直到程序退出为止。

\bugbox{nemu/src/monitor/cpu-exec.c}{%
  虽然一般情况下程序的指令数量都不会超过最大整数的值,但如果超过了那么就会发生
  错误。为了解决这个可能会出现的问题,我们可以在\code{cpu-exec()}函数中做一点修改,
  只要让当参数值为最大整数(也就是-1)时一直运行即可,可以把\code{for}循环中更新
  部分的\code{n-\/-}变为\code{(n == (uint64\_t)-1) ? n : n-\/-}。
}

继续观察\code{cpu-exec()}函数,该函数使用了状态机来判断虚拟机的运行状态,我们可以
先不管。函数内主体是一个\code{for}循环,在循环体内首先记录下旧的\pc 值,之后
调用\code{exec\_once()}函数,从这个函数的名字和出现的位置我们可以推断出来这个函数
的功能是让虚拟机执行下一条指令。执行了一条指令之后就继续进入下一个循环(可以先忽略
掉条件编译部分)。

接下来我们就需要去看看\code{exec\_once()}函数了。这个函数在文件\file{nemu/src/cpu/cpu.c}
中定义,函数体也比较简单,先调用\code{isa\_exec()}函数再调用了\code{update\_pc()}函数。
由于NEMU代码框架中定义了三种指令集架构,对于每一种架构来说其执行指令的过程都是不同
的,所以能够看出来\code{isa\_exec()}函数的存在是为了屏蔽掉不同架构带来的差异,
相当于一个接口,不同的架构只需要实现这个接口就可以达到架构无关的目的。\code{update\_pc()}
函数从名字上就能看出来是更新指令计数器\pc 的值。

下面我们就要进入到\code{isa\_exec()}函数内看看了。由于这个接口对于不同的架构来说
实现是不同的,因为本次选择的架构是\arch,所以我们就要去看\arch 下的实现。打开文件
\file{\archpath/exec/exec.c},在文件的末尾定义了\code{isa\_exec()}函数。在该函数体
中,首先通过\code{instr\_fetch()}函数取出一个字节的opcode,并设置好\code{decinfo}
全局译码结构相关的成员,之后调用了一次\code{set\_width()}函数,最后调用\code{idex()}
函数。

看到这里你可能会感觉有点懵,按理来说让CPU执行一条指令应该先从\pc 处取出一条指令,
之后对这个指令进行解码执行即可,为什么这里取出指令的长度只有一个字节?这里的
\code{set\_width()}函数又是有什么作用呢?要解决这个问题就要涉及到\arch 架构指令
的编码方式了。

\subsection{\arch 架构指令编码}
现在我们已经根据\emph{c}命令的执行过程来到了\arch 架构的指令执行函数,为了继续进行
我们需要知道\arch 架构指令的编码方式。

\arch 的指令编码格式如图\ref{fig:isa-instr-format}所示,每个域的作用如表\ref{tab:x86-instr-format}所示。

\hustfigure[0.8\textwidth]{figure/x86-instruction-format.pdf}{\arch 指令编码格式}{fig:isa-instr-format}

\begin{table}[!htpb]
  \centering
  \caption{\arch 指令域及其作用}
  \label{tab:x86-instr-format}
  \begin{tabular}{ccc}
    \toprule
    \textbf{域名}              & \textbf{作用}                      & \textbf{字节数}    \\
    \midrule
    instruction prefix  & 改变指令行为 (如\code{REP}) & 0或1        \\
    address-size prefix & 改变地址位宽                & 0或1        \\
    operand-size prefix & 改变操作数大小              & 0或1        \\
    segment override    & 改变寻址使用的段寄存器      & 0或1        \\
    opcode              & 指令操作码                  & 1或2        \\
    modr/m              & \multirow{2}*{指定寻址方式} & 0或1        \\
    sib                 &                             & 0或1        \\
    displacement        & 内存寻址偏移                & 0, 1, 2或4  \\
    immediate           & 立即数                      & 0, 1, 2或4  \\
    \bottomrule
  \end{tabular}
\end{table}

能够看到在编码格式中包含了很多部分,并且前面有很多部分都是可有可无的。

我们先从简单的来说起。在这些域中,必须存在的就是\emph{opcode}部分。\emph{opcode}
可以理解为operation code,也就是操作码,它代表了这条指令的功能,可以说是指令的
id,不同的指令其\emph{opcode}是不同的。举一个最简单的例子,对于\arch 指令
\code{PUSH EAX}来说,该指令仅有一个字节也就是\emph{opcode},为\emph{0x50}。
你可能会问,\emph{opcode}域不是可以有两个字节吗?我们怎么在译码的时候知道这个\emph{opcode}
是一个字节还是两个字节呢?事实上,由于一个字节只有256个值,而对于\arch 这样的CISC
架构是不够的,为了能够拥有更多的指令,\arch 采用了变长编码,将\code{0x0F}作为转义
字节,也就是说当第一个\emph{opcode}是\code{0x0F}是就代表这个\emph{opcode}有两位,
类似于赫夫曼编码。比如,\code{SETO}指令是当OF标志位为1时将对应的操作符设为1,这个
指令的\emph{opcode}是\code{0x0F 0x90}。

之后我们来看看\emph{opcode}之前的域。\emph{opcode}之前的域都是可有可无的,当这些
域中的某一个存在时,就会改变这条指令的行为。比如如果存在operand-size prefix域,
那么原本一条指令操作数的位宽就会从32位变为16位。这样就会有一个问题,当我们进行
译码时,怎么知道这些域存不存在呢?事实上\emph{opcode}前面的这些域与\emph{opcode}
包含的所有编码都是两两不同的,比如operand-size prefix域要是存在的话只能
是\code{0x66},而\emph{opcode}域的编码并不包含\code{0x66},所以在译码时就能够根据
当前这个字节的值来判断这个字节是代表着\emph{opcode}域还是其之前的某个域。上面指出
了\code{PUSH EAX}指令的编码为\code{0x50},而我们又知道\code{0x66}代表着operand-size
prefix域,所以编码\code{0x66 0x50}就代表着\code{PUSH AX},即将操作数从32位的\code{EAX}
变为了16位的\code{AX}。

在\emph{opcode}之后是\emph{modr/m}和\emph{sib}域。这两个域可以说是\arch 编码的
精华,在这两个域中包含了以下信息:
\begin{itemize}
  \item 寻址方式或是要使用的寄存器
  \item 需要用到的寄存器,或是指定指令行为的信息
  \item 基址、变址以及比例因子信息
\end{itemize}

总的来说,这两个域说明了这个指令要怎样去找到操作数。比如对于\code{MOV}指令,这个
指令的功能是将数据从一个地方复制到另一个地方,操作数可能在两个寄存器中,可能都在
内存中,也可能一个在寄存器一个在内存中。\emph{modr/m}和\emph{sib}域就指定了这条
\code{MOV}指令的两个操作数到底在哪,如果在寄存器中会说明在哪个寄存器,如果是在
内存中会说明内存的寻址方式以及寻址的基址、变址寄存器等信息。

\emph{modr/m}和\emph{sib}各自都有8位,他们各自又将这8位划分成了三个部分,如图\ref{fig:modrm-sib}
所示。

\hustfigure[0.5\textwidth]{figure/modrm-sib.pdf}{\emph{modr/m}和\emph{sib}域的划分}{fig:modrm-sib}

% TODO
对于\emph{modr/m}和\emph{sib}的具体编码详见\cite{i386-manual}。

最后还有\emph{displacement}和\emph{immediate}域,这两个域比较简单,\emph{displacement}
域代表了内存寻址时的偏移,要和\emph{modr/m}以及\emph{sib}配合使用。而\emph{immediate}
域则是代表了立即数,是否存在该域要看具体的\emph{opcode}。

\subsection{\arch 指令译码与执行过程}
在大致知道了\arch 指令的编码方式之后,我们就能够进行译码的实现。再次回到\code{isa\_exec()}
函数,在函数体中首先在\pc 处取出一个字节,这时我们大概能够推断出取出的这一个字节
是\emph{opcode}(当然也有可能是\emph{opcode}前面的那些域,我们后面再来说明如果是
前缀域要怎么办),取出了\emph{opcode}之后调用了一次\code{set\_width()}函数,从函数名字
来看这个函数是设置宽度,能够推断出这个函数的大概作用是设置操作数的宽度,我们先忽略
掉,之后就调用了\code{idex()}函数。\code{idex()}在名字上能够看出来是id和ex,其中
id指的是instruction decode,即指令译码,ex指的是execute,也就是指令执行。那么\code{idex()}
的作用我们能推断出是译码以及执行的意思了。

虽然大概知道了\code{isa\_exec()}函数的执行流程,但还是很不清晰,还是不了解\code{set\_width()}
的具体实现,以及译码时\emph{opcode}之前的域和之后的域都是怎么推断出来的。

为了对译码部分的实现有更清晰的认识,我们进入到\code{idex()}里面看看其是怎么执行的。
\code{idex()}函数在\file{nemu/include/cpu/exec.h}中定义,函数也非常简单,有两个
参数,一个是\pc 我们很好理解,另一个是\code{OpcodeEntry}结构指针。在函数体内先判断
\code{OpcodeEntry}是否存在一个名为\code{decode}的函数指针,如果存在则调用该函数。
之后调用\code{OpcodeEntry}下名为\code{execute()}的函数。也就是说,在\code{idex()}
函数中只是简单地调用了这个指令对应的\code{decode()}函数以及\code{execute()}函数。
这里比较重要的是\code{OpcodeEntry}这个参数,能够看到这个参数中包含了对应指令的
译码及运行的实现函数指针。我们回到\code{isa\_exec()}函数,发现这个\code{OpcodeEntry}
参数传入的值是\code{\&opcode\_table[opcode]}。那么我们就看看\code{opcode\_table}是
一个什么数组。进入文件\file{\archpath/exec/exec.c},能够看到在这个文件里用很长的
代码定义了静态数组\code{opcode\_table},这个数组的类型就是\code{OpcodeEntry}。
而\code{OpcodeEntry}结构在\code{nemu/include/cpu/exec.h}中定义,其定义为:
\begin{codes*}{C}
typedef void (*EHelper) (vaddr_t *);
typedef void (*DHelper) (vaddr_t *);
typedef struct {
  DHelper decode;
  EHelper execute;
  int width;
} OpcodeEntry;
\end{codes*}

能够看到该结构只有三个成员,一个是\code{decode}函数指针,一个是\code{execute}函数
指针,另一个是\code{width}。根据\code{OpcodeEntry}的名字以及其成员和出现的位置
我们能够知道,这个结构的含义是指令译码和执行的入口。由于\arch 指令是变长编码,
不同指令有不同的宽度,并且不同的指令有不同的功能,所以我们在对指令进行译码和执行
时就不能一概而论,必须要根据具体指令来选择其对应的译码和执行的实现,而区分指令
则是根据\emph{opcode}域,所以当我们得到了一条指令的\emph{opcode}之后就需要转入
该\emph{opcode}对应指令的译码和执行函数。\code{opcode\_table}就是保存所有\emph{opcode}
对应实现函数的表,这个表一共有512个成员,前256个代表了\emph{opcode}长度为1的那些
指令,后256个代表了\emph{opcode}长度为2的那些指令。

由于每个\emph{opcode}对应的实现是已知的,所以\code{opcode\_table}表是静态的,
应在编译时就设置好其成员,也就是说需要在代码文件中填写好。为了方便填写
\code{opcode\_table}表,在\file{nemu/include/cpu/exec.h}中定义了一些宏,如下所示。
\begin{codes*}{C}
#define IDEXW(id, ex, w) {concat(decode_, id), concat(exec_, ex), w}
#define IDEX(id, ex)     IDEXW(id, ex, 0)
#define EXW(ex, w)       {NULL, concat(exec_, ex), w}
#define EX(ex)           EXW(ex, 0)
#define EMPTY            EX(inv)
\end{codes*}

这些宏最主要的一个就是\code{IDEXW},这个宏有三个参数,其中\code{id}代表了译码函数名,
\code{ex}代表了执行函数名,\code{w}代表了\code{OpcodeEntry}中的\code{width}成员。
\code{IDEX}宏与其类似,只不过将\code{width}成员设为0。\code{EXW}宏是将译码实现
函数设为\code{NULL},\code{EX}宏是仅设置执行函数,\code{EMPTY}宏的定义是\code{EX(inv)},
而\code{inv}代表的是invalid,也就是说\code{EMPTY}代表这个指令还未实现。
在刚开始时,表项中大多数都是\code{EMPTY},也就是还未实现,需要自己去增加。

看到这可能还会有一个问题,为什么有些指令的宽度和译码函数没有呢?指令中的宽度代表的
是什么意思呢?我们先说为什么有些指令的译码函数没有。由于\arch 是变长指令,所以
不同指令的长度不同,从指令的格式上来看\emph{opcode}域之后的那些(比如\emph{modr/m}
和\emph{sib})都是可有可无的,并且当他们存在时字节数也不一定。其实\emph{opcode}域
之后的那些域都是由\emph{opcode}决定的。\emph{opcode}决定了具体的指令,而具体的指令
又能决定操作数的信息,而\emph{opcode}后面的那些域正是代表着操作数的信息,所以当
\emph{opcode}知道了之后就知道是否需要后面的某些域来确定操作数信息。那么\code{OpcodeEntry}
中的\code{width}成员代表着什么意思呢?这个成员代表的是指令操作数的长度,有些
\emph{opcode}决定了这条指令操作数的长度,这时我们就需要在\code{OpcodeEntry}中设置,
而有些则并不确定。还记得指令格式中的\emph{operand-size prefix}域吗?有些指令靠他
来确定操作数的位宽,当没有这个前缀时位宽是32,当存在时位宽是16。需要知道指令操作数
位宽的原因是我们在取操作数时需要知道,比如在内存中取操作数时,操作数字节数是0、1、
2或4时取出来的值是不同的(当然在极个别情况下可能会相同)。这时我们就能够解释\code{isa\_exec()}
函数中为什么要调用\code{set\_width()}函数了。\code{set\_width()}函数就是根据是否
存在\emph{operand-size prefix}域来设置操作数的位宽到底是32还是16。

最后还有一个问题,那就是\emph{opcode}之前的域到底是怎么处理的?我们在前面假设指令
取出的第一个字节就是\emph{opcode},但如果在\emph{opcode}之前有其他的域,比如
\emph{operand-size prefix}域,那要怎么办呢?事实上,由于前面提到\emph{opcode}的
编码与\emph{opcode}之前的域的编码是两两不同的,所以在NEMU中也将\emph{opcode}之前的
域当作\emph{opcode}处理了,他们在\code{opcode\_table}中也存在着表项,并且这个
表项中只包含执行函数。而在其执行函数中,会在\code{decinfo}这个全局译码信息中记录
下这个前缀域要改变的指令行为,之后会再次调用\code{isa\_exec()}函数。注意这里的调用
并不代表是进入到下一个指令的执行,而是重新开始这一条指令的译码及执行。由于在前缀域
的执行函数中已经记录下了这次执行需要改变的行为,所以在真正指令译码和执行的过程中
就可以通过\code{decinfo}来判断是否需要改变某些行为,比如当存在\emph{operand-size prefix}
时译码过程中在取操作数时会将操作数的位宽变为16。

由于NEMU比较简单,所以在所有的指令中只会出现\emph{operand-size prefix}这一种前缀域,
该域的执行函数在\file{\archpath/exec/prefix.c}中定义,如下所示。
\begin{codes*}{C}
// 先忽略掉make_EHelper,这里主要关注函数体
make_EHelper(operand_size) {
  decinfo.isa.is_operand_size_16 = true;
  isa_exec(pc);
  decinfo.isa.is_operand_size_16 = false;
}
\end{codes*}

好了,现在总结一下,在NEMU中,\arch 架构在执行一条指令时,会先在\pc 处取出一个字节,
这个字节就是这一条指令的\emph{opcode}(前面说了如果是前缀域则也会按照\emph{opcode}
一样处理),然后以\emph{opcode}作为下标从\code{opcode\_table}中取出这一条指令对应
的入口,之后如果入口中定义了译码函数则会调用译码函数,在译码函数中会根据这一条
指令的编码规定去取出\emph{opcode}后面的域,并且根据这些域以及指令的内容去取出
操作数,并设置目的操作数的信息。之后会调用入口中的执行函数,对译码阶段中设置的
源操作数进行运算,并将结果放入目的操作数中。
